\chapter{Specifikacija programske potpore}

\section{Funkcionalni zahtjevi}

\noindent \textbf{Dionici:}

\begin{packed_enum}
	
	\item Donor		
	\item Djelatnik banke	
	\item Administrator
	\item Razvojni tim
	
\end{packed_enum}

\noindent \textbf{Aktori i njihovi funkcionalni zahtjevi:}


\begin{packed_enum}
	\item  \underbar{Neregistrirani/neprijavljeni korisnik(inicijator) može:}
	
	\begin{packed_enum}
		
		\item pregledati trenutno stanje zaliha
		\item se registrirati u sustav, stvoriti korisnički račun za koji su mu potrebni matični i kontakt podaci
		
	\end{packed_enum}
	
	\item  \underbar{Donor (inicijator) može:}
	
	\begin{packed_enum}
		
		\item pregledavati i mijenjati osobne podatke
		\item pregledavati povijest svojih doniranja
		\item iz aplikacije dobiti PDF potvrdu
		\item pregledati poruku u ovisnosti o trenutnom stanju zaliha krvi
		
	\end{packed_enum}
	
	\item  \underbar{Djelatnik banke (inicijator) može:}
	
	\begin{packed_enum}
		
		\item kreirati korisnički profil donora
		\item evidentirati svaki pokušaj doniranja (uspješan / neuspješan) 
		\item evidentirati privremeno ili trajno odbijanje
		\item evidentirati potrošnju krvi 
		\item aktivirati račun aktivacijskim linkom i odabrati lozinku
		\item vidjeti popis registriranih donora
		
	\end{packed_enum}
	
	\item  \underbar{Administrator (inicijator) može:}
	
	\begin{packed_enum}
		
		\item definirati gornju i donju granicu optimalne količine krvi
		\item kreirati nove korisničke račune za ulogu djelatnika banke
		\item deaktivirati korisnički račun djelatnika banke ili donora
		\item vidjeti popis registriranih svih korisnika i njihovih osobnih podataka
		
	\end{packed_enum}
	
	\item  \underbar{Baza podataka (sudionik):}
	
	\begin{packed_enum}
		
		\item pohranjuje sve podatke o korisnicima i njihovim ovlastima
		\item pohranjuje trenutno stanje količine krvi,te donju i gornju granicu optimalne količine krvi
		
	\end{packed_enum}
\end{packed_enum}

\eject 



\subsection{Obrasci uporabe}

\subsubsection{Opis obrazaca uporabe}

\noindent \underbar{\textbf{UC$11$ -\textit{Aktivacija računa}\eject	}}
\begin{packed_item}
	
	\item \textbf{Glavni sudionik: }\textit{Korisnik, djelatnik banke}\eject
	\item  \textbf{Cilj:} \textit{Aktivirati račun}\eject
	\item  \textbf{Sudionici:}\textit{Baza podataka}\eject 
	\item  \textbf{Preduvjet:}\textit{Registracija}\eject
	\item  \textbf{Opis osnovnog tijeka:}
	
	\item[] \begin{packed_enum}
		
		\item \textit{Korisnika nakon klika na link u mailu vodi u aplikaciju}\eject
		\item \textit{Prikazuje se poruka o uspješnoj registraciji i traži se odabir lozinke}\eject 
		\item \textit{Korisnik odabire lozinku}\eject 
		\item \textit{Korisnik se automatski prijavljuje}\eject
		
	\end{packed_enum}
	
\end{packed_item}

\noindent \underbar{\textbf{UC$12$ -\textit{Evidentiranje pokušaja doniranja}\eject	}}
\begin{packed_item}
	
	\item \textbf{Glavni sudionik: }\textit{Djelatnik banke}\eject
	\item  \textbf{Cilj:} \textit{Evidentirati pokušaj doniranja}\eject
	\item  \textbf{Sudionici:}\textit{Baza podataka}\eject 
	\item  \textbf{Preduvjet:}\textit{Registracija}\eject
	\item  \textbf{Opis osnovnog tijeka:}
	
	\item[] \begin{packed_enum}
		
		\item \textit{Djelatnik odabere opciju za evidentiranje pokušaja doniranja određenog donora}\eject
		\item \textit{Djelatnik evidentira s uspješan ili neuspješan pokušaj}\eject 
		\item \textit{Djelatnik banke sprema promjene}\eject 
		\item \textit{Baza podataka se ažurira, povećava se razina određene vrste krvi i zapisuje se pokušaj doniranja krvi}\eject
		
	\end{packed_enum}
	
\end{packed_item}
\noindent \underbar{\textbf{UC$13$ -\textit{Brisanje donora}\eject	}}
\begin{packed_item}
	
	\item \textbf{Glavni sudionik: }\textit{Administrator}\eject
	\item  \textbf{Cilj:} \textit{Izbrisati kor. račun donora}\eject
	\item  \textbf{Sudionici:}\textit{Baza podataka}\eject 
	\item  \textbf{Preduvjet:}\textit{Administrator prijavljen}\eject
	\item  \textbf{Opis osnovnog tijeka:}
	
	\item[] \begin{packed_enum}
		
		\item \textit{Administrator odabire opciju Prikaz donora}\eject
		\item \textit{Administratoru se pokažu registrirani donori}\eject 
		\item \textit{Administrator briše donora}\eject 
		
	\end{packed_enum}
	
\end{packed_item}

\noindent \underbar{\textbf{UC$14$ -\textit{Definiranje optimalnih granica}\eject	}}
\begin{packed_item}
	
	\item \textbf{Glavni sudionik: }\textit{Administrator}\eject
	\item  \textbf{Cilj:} \textit{Definirati gornju i donju granicu optimalnih količina krvi za pojedinu grupu}\eject
	\item  \textbf{Sudionici:}\textit{Baza podataka}\eject 
	\item  \textbf{Preduvjet:}\textit{Administrator je prijavljen}\eject
	\item  \textbf{Opis osnovnog tijeka:}
	
	\item[] \begin{packed_enum}
		
		\item \textit{Administrator odabire opciju Definiraj granice}\eject
		\item \textit{Prikazuju se trenutno određene granice za pojedinu vrstu krvi}\eject 
		\item \textit{Administrator definira granice za pojedinu vrstu krvi}\eject 
		\item \textit{Administrator sprema promjene}\eject
		
	\end{packed_enum}
\end{packed_item}

\noindent \underbar{\textbf{UC$15$ -\textit{Pregled popisa djelatnika}\eject	}}
\begin{packed_item}
	
	\item \textbf{Glavni sudionik: }\textit{Administartor}\eject
	\item  \textbf{Cilj:} \textit{Prikazati popis svih djelatnika banke}\eject
	\item  \textbf{Sudionici:}\textit{Baza podataka}\eject 
	\item  \textbf{Preduvjet:}\textit{Administrator je prijavljen}\eject
	\item  \textbf{Opis osnovnog tijeka:}
	
	\item[] \begin{packed_enum}
		
		\item \textit{Administrator odabire opciju Prikaz popisa djelatnika banke}\eject
		\item \textit{Prikazuje se popis svih djelatnika banke po abecednom redu}\eject 
		
\end{packed_item}
\noindent \underbar{\textbf{UC$16$ -\textit{Brisanje djelatnika banke}\eject	}}
\begin{packed_item}
	
	\item \textbf{Glavni sudionik: }\textit{Administrator}\eject
	\item  \textbf{Cilj:} \textit{Izbrisati kor. račun djelatnika banke}\eject
	\item  \textbf{Sudionici:}\textit{Baza podataka}\eject 
	\item  \textbf{Preduvjet:}\textit{Administrator prijavljen}\eject
	\item  \textbf{Opis osnovnog tijeka:}
	
	\item[] \begin{packed_enum}
		
		\item \textit{Administrator odabire opciju Prikaz djelatnika banke}\eject
		\item \textit{Administratoru se pokažu registrirani djelatnici banke}\eject 
		\item \textit{Administrator briše djelatnika banke}\eject 
		
	\end{packed_enum}
	
\end{packed_item}

\noindent \underbar{\textbf{UC$17$ -\textit{Kreiranje kor. računa djelatnika banke}\eject	}}
\begin{packed_item}
	
	\item \textbf{Glavni sudionik: }\textit{Administrator}\eject
	\item  \textbf{Cilj:} \textit{Kreirati računa djelatnika banke}\eject
	\item  \textbf{Sudionici:}\textit{Baza podataka}\eject 
	\item  \textbf{Preduvjet:}\textit{Administrator je prijavljen}\eject
	\item  \textbf{Opis osnovnog tijeka:}
	
	\item[] \begin{packed_enum}
		
		\item \textit{Administrator odabire opciju "Kreiraj korisnički račun djelatnika banke"}\eject
		\item \textit{Prikazuje se forma koju popunjava}\eject 
		\item \textit{Odabire opciju "Kreiraj račun"}\eject 
		\item \textit{Baza podataka se ažurira}\eject
		\item \textit{Šalje se aktivacijski link na email djelatnika banke}\eject
	\end{packed_enum}
	
	\item  \textbf{Opis mogućih odstupanja:}
	
	\item[] \begin{packed_item}
		
		\item[2.a] $<$opis mogućeg scenarija odstupanja u koraku 2$>$
		\item[] \begin{packed_enum}
			
			\item $<$opis rješenja mogućeg scenarija korak 1$>$
			\item $<$opis rješenja mogućeg scenarija korak 2$>$
			
		\end{packed_enum}
		\item[2.b] $<$opis mogućeg scenarija odstupanja u koraku 2$>$
		\item[3.a] $<$opis mogućeg scenarija odstupanja  u koraku 3$>$
		
	\end{packed_item}
\end{packed_item}



\subsubsection{Dijagrami obrazaca uporabe}

\textit{Prikazati odnos aktora i obrazaca uporabe odgovarajućim UML dijagramom. Nije nužno nacrtati sve na jednom dijagramu. Modelirati po razinama apstrakcije i skupovima srodnih funkcionalnosti.}
\eject		

\subsection{Sekvencijski dijagrami}

\textbf{\textit{dio 1. revizije}}\\

\textit{Nacrtati sekvencijske dijagrame koji modeliraju najvažnije dijelove sustava (max. 4 dijagrama). Ukoliko postoji nedoumica oko odabira, razjasniti s asistentom. Uz svaki dijagram napisati detaljni opis dijagrama.}
\eject

\section{Ostali zahtjevi}

\textbf{\textit{dio 1. revizije}}\\

\textit{Nefunkcionalni zahtjevi i zahtjevi domene primjene dopunjuju funkcionalne zahtjeve. Oni opisuju \textbf{kako se sustav treba ponašati} i koja \textbf{ograničenja} treba poštivati (performanse, korisničko iskustvo, pouzdanost, standardi kvalitete, sigurnost...). Primjeri takvih zahtjeva u Vašem projektu mogu biti: podržani jezici korisničkog sučelja, vrijeme odziva, najveći mogući podržani broj korisnika, podržane web/mobilne platforme, razina zaštite (protokoli komunikacije, kriptiranje...)... Svaki takav zahtjev potrebno je navesti u jednoj ili dvije rečenice.}





