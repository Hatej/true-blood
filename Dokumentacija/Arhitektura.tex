\chapter{Arhitektura i dizajn sustava}
		
		Arhitekturu našeg sustava možemo podijeliti na tri podsustava:	
		\begin{itemize}
		\item 	\textit{Web preglednik}
		\item 	\textit{Baza podataka}
		\item 	\textit{Web poslužitelj}		
		\end{itemize}
		
		\begin{figure}[H]
			\centering
			\includegraphics[width=\textwidth, scale=2.0]{slike/arhitektura.png}
			\caption{Arhitektura našeg sustava}
			\label{fig:arhitektura}
		\end{figure}
\eject
	
	
	
	\underbar{ \textit{Web preglednik} }{je program koji korisniku omogućuje pregled web-stranica i multi-medijalnih sadržaja vezanih uz njih. Korisnik putem web preglednika šalje zahtjeve na obradu poslužitelju.}

	\underbar{ \textit{Baza podataka}}{ je zbirka zapisa pohranjenih u računalu na sustavan način, tako da joj se računalni program može obratiti prilikom odgovaranja na problem. Web poslužitelj komunicira s bazom podataka te povlači potrebne zapise iz nje.}
	
	\underbar{ \textit{Web poslužitelj}}{ je srce našeg sustava. Njegova zadaća je komunikacija klijenta s aplikacijom te bazom podataka. Pri komunikaciji koristi HTTP protokol.Korisnik i aplikacija razmjenjuju HTTP zahtjeve (eng. HTTP request) i HTTP odgovore (HTTP response). Radi jednostavnosti, baza podataka je također smještena na poslužitelju.}\\
	Korisnik kroz grafičko sučelje, odnosno prednji kraj, šalje zahtjeve na REST pristupne točke stražnjeg kraja. Tada stražnji kraj procesuira zahtjev i ako je potrebno komunicira s bazom podataka. Nakon konstrukcije, stražnji kraj šalje odgovor prednjem kraju u obliku JSON objekta, a prednji kraj procesuira odgovor i promjene prikazuje korisniku u obliku HTML stranice.
	Za aplikaciju je odabrana višeslojna arhitektura temeljena na \textbf{MVC} (Model - View - Controller) arhitekturnom stilu te uslužnoj arhitekturi. Podjela slojeva možemo napraviti na idući način:
	\begin{figure}[H]
			\centering
			\includegraphics[scale=0.9]{slike/arhitektura2.png}
			\caption{Podjela slojeva}
			\label{fig:slojevi}
		\end{figure}
	\begin{itemize}
\eject
		\item 	\textit{sloj korisničke strane} - korisničko sučelje implementirano u Reactu
		\item 	\textit{sloj nadglednika} - REST nadglednici
		\item 	\textit{sloj domene}	- model podataka iz domene primjene
		\item 	\textit{sloj za pristup podacima} - posrednik između sloja domene i baze podataka
		\item 	\textit{sloj baze podataka} - pohrana podataka	
		\end{itemize}
	{Ovakva arhitektura odabrana je zbog poželjnih svojstava MVC arhitekturnog stila i višeslojne arhitekture: razvoj pojedinih slojeva je jednostavniji i u velikom stupnju nezavisan od razvoja drugih slojeva. Također komunikacija prednjeg i stražnjeg kraja ostvarena je primjenom REST arhitekturnog stila. Zbog toga su i prednji i stražnji kraj neovisni jeziku implementacije, što potiče ponovnu uporabu. }
	
	 {Model-View-Controller se sastoji od:}
	\begin{itemize}
		\item 	\textbf{Model} {je centralni dio aplikacije, koja obuhvaća promjenljivu (dinamičku) strukturu podataka, direktno upravljanje podacima, logikom i pravilima aplikacije}
		\item 	\textbf{View}{ je bilo koji izlazni prikaz podataka u korisničkom okruženju, pri čemu se isti podaci mogu prikazati na više načina}
		\item 	\textbf{Controller} {ulazne podatke pretvara u komande koje upravljaju modelom ili prikazom podataka u korisničkom okruženju}
	\end{itemize}
	
		\begin{figure}[H]
			\centering
			\includegraphics[width=100mm, scale=0.1]{slike/MVC.jpeg}
			\caption{MVC model}
			\label{fig:arhitektura}
		\end{figure}
\eject

		

				
		\section{Baza podataka}
			
			
			
		Za potrebe našeg sustava koristit ćemo relacijsku bazu podataka koja svojom strukturom olakšava modeliranje stvarnog svijeta. Gradivna jedinka baze je relacija, odnosno tablica koja je definirana svojim imenom i skupom atributa. Zadaća baze podataka je brza i jednostavna pohrana, izmjena i dohvat podataka za daljnju obradu.
Baza podataka ove aplikacije sastoji se od sljedećih entiteta: 
\begin{itemize}
		\item korisnikAplikacije
		\item uloge
		\item pokusajDoniranja
		\item krvnaVrsta
		\item potrosnjaKrvi
		\item zdravstveniPodaci
		\item doniranjeZdravljeOdgovori
		
	\end{itemize}

		\eject
			\subsection{Opis tablica}
			

				\textbf{korisnikAplikacije \textit{•}}
				 Ovaj entitet predstavlja sve korisnike aplikacije (admin, djelatnik, donor) i diferencira ih pomoću ulogaId što je referenca na tablicu uloga. KorisnikId i lozinka se koristi za login u aplikaciju. Za admina i djelatnika postoje atributi : ime, prezime, oib, email dok su ostale vrijednosti null. Kod stvaranja atributa trajnoOdbijanjeDarivanja je false, a razlogOdbijanja je null ,ako se korisniku odbije darivanje zauvijek onda se ti atributi mjenjaju.

				
				\begin{longtblr}[
					label=none,
					entry=none
					]{
						width = \textwidth,
						colspec={|X[15,l]|X[6, l]|X[20, l]|}, 
						rowhead = 1,
					} %definicija širine tablice, širine stupaca, poravnanje i broja redaka naslova tablice
					\hline \multicolumn{3}{|c|}{\textbf{korisnikAplikacije}}	 \\ \hline[3pt]				
					\SetCell{LightGreen}korisnikId & VARCHAR & id pomoću kojeg se korisnik prijavljuje u sustav (jednako donorId za donore) -> generira se pomoću inicijala i zadnjih 5 znamenki oiba npr.
					Ivica ivic 0303041001340 ima korisnikId = ii01340\\ \hline
					lozinka	& VARCHAR &  lozinka za login 	\\ \hline 
					ime & VARCHAR	&  ime korisnika		\\ \hline 
					prezime & VARCHAR	& prezime korisnika	\\ \hline 
					mjestoRodenja & VARCHAR & mjesto gdje se korisnik rodio (nullable) \\ \hline
					oib & CHAR(11) & oib korisnika \\ \hline
					adresaStanovanja & VARCHAR & adresa na kojoj korisnik stanuje (nullable) \\ \hline
					mjestoZaposlenja & VARCHAR & firma u kojoj je korisnik zaposlen (nullable) \\ \hline
					email & VARCHAR & email na koji korisniku dolaze korisne informacije (nullable) \\ \hline 
					brojMobitelaPrivatni & VARCHAR	&  broj na koji korisniku dolaze korisne infromacije privatni	(nullable)	\\ \hline
					brojMobitelaPoslovni & VARCHAR	&  broj na koji korisniku dolaze korisne infromacije poslovni ( neobavezan za sve uloge) (nullable)		\\ \hline
                     datumRodenja & DATE &  datum rođenja	\\ \hline
                     \SetCell{LightBlue}krvId & INT & vrsta krvi donora \\ \hline
                     trajnoOdbijenoDarivanje & Boolean &  (nullable)\\ \hline
                     
					 \SetCell{LightBlue}ulogaId & INT &  označava je li korisnik admin, djelatnik ili donor \\ \hline 
					aktivacijskiKljuc & VARCHAR(10) & aktivacijski ključ koji služi za aktivaciju računa, nakon aktivacije NULL \\
\hline		
					\end{longtblr}
				
				\textbf{uloge \textit{•}}
				entitet koji sadrži dva atributa, id za oznaku rednog broja uloge i ulogaName za ime 						uloge
				\begin{longtblr}[
					label=none,
					entry=none
					]{
						width = \textwidth,
						colspec={|X[6,l]|X[6, l]|X[20, l]|}, 
						rowhead = 1,
					} %definicija širine tablice, širine stupaca, poravnanje i broja redaka naslova tablice
					\hline \multicolumn{3}{|c|}{\textbf{Uloge}}	 \\ \hline[3pt]
					\SetCell{LightGreen}ulogaId & INT	& označava id uloge (admin, djelatnik ili donor)\\ \hline
					ulogaName	& VARCHAR & ime uloge(donor,admin...)  	\\ \hline 
					
				\end{longtblr}
	\eject
				\textbf{pokusajDonacije\textit{•}}
				-entitet koji služi za spremanje pokušaja doniranja, sprema podatke o datumu, mjestu, darivatelju, djelatniku, uspješnosti i razlogu odbijanja ako je uspješnost false
				
				\begin{longtblr}[
					label=none,
					entry=none
					]{
						width = \textwidth,
						colspec={|X[15,l]|X[6, l]|X[20, l]|}, 
						rowhead = 1,
					} %definicija širine tablice, širine stupaca, poravnanje i broja redaka naslova tablice
					\hline \multicolumn{3}{|c|}{\textbf{pokusajDonacije}}	 \\ \hline[3pt]
					\SetCell{LightGreen}brDoniranja & VARCHAR & redni broj donacije\\ \hline
					datum & DATE & datum donacije \\ \hline
					mjestoDarivanja	& VARCHAR & opisno mjesto donacije	\\ \hline 
					\SetCell{LightBlue}korisnikIdDjelatnika & VARCHAR & korisnikId od djelatnika \\ \hline
					\SetCell{LightBlue}korisnikId & VARCHAR & korisnikId od donora \\ \hline

					uspjeh	& boolean & true za uspjeh, false za neuspjeh  	\\ \hline 			
					
				\end{longtblr}
				
				\textbf{krvnaVrsta\textit{•}}
				
				- entitet čuva podatke o zalihi i predviđenim gornjim i donjim granicama za sve krvnih grupa
				\begin{longtblr}[
					label=none,
					entry=none
					]{
						width = \textwidth,
						colspec={|X[15,l]|X[6, l]|X[20, l]|}, 
						rowhead = 1,
					} %definicija širine tablice, širine stupaca, poravnanje i broja redaka naslova tablice
					\hline \multicolumn{3}{|c|}{\textbf{krvnaVrsta}}	 \\ \hline[3pt]
					\SetCell{LightGreen}krvId & SERIAL & ID krvi serial \\ \hline
					imeKrvneGrupe & VARCHAR & (A+,A-,AB+,B-...) \\ \hline
					gornjaGranica & INT & gornja granica dopuštene količine krvi u jedinicama\\ \hline

					donjaGranica	& INT & donja granica dopuštene količine krvi u jedinicama  	\\ \hline 
					trenutnaZaliha	& INT &  trenutna zaliha konkretne krve grupe u jedinicama 	\\ \hline 
					
					
				\end{longtblr}
\eject			
				
				\textbf{potrosnjaKrvi\textit{•}}
				
				entitet koji prati isporuke krvi
				\begin{longtblr}[
					label=none,
					entry=none
					]{
						width = \textwidth,
						colspec={|X[15,l]|X[6, l]|X[20, l]|}, 
						rowhead = 1,
					} %definicija širine tablice, širine stupaca, poravnanje i broja redaka naslova tablice
					\hline \multicolumn{3}{|c|}{\textbf{potrosnjaKrvi}}	 \\ \hline[3pt]
					\SetCell{LightGreen} idPotrosnje & SERIAL & redni broj isporuke krvi \\ \hline
					timestampPotrosnje & timestamp & timestamp potrošnje \\ \hline
					\SetCell{LightBlue}krvId & INT & id krvne grupe \\ \hline
					količinaJedinica & INT &  broj jedinica koje su se potrošili \\ \hline 
					
					\SetCell{LightBlue}korisnikIdDjelatnika & VARCHAR &  korisničko ime djelatnika koji je inicirao slanje krvi bolnici	\\ \hline 
					lokacijaPotrosnje & VARCHAR & opisna lokacija kamo ide isporuka krvi \\ \hline
					
				\end{longtblr}
				
				\textbf{zdravstveniPodaci\textit{•}}
				
				entitet koji sprema sve moguće zdravstvene podatke koji se pitaju korisnika u upitniku. Oni sadrže svoj id, koji se sam generira u tablici, opis zdravstvenog podatka i težinu kriterija odnosno, 0 ako se na temelju toga podatka trajno odbija darivanje i 1 ako se privremeno odbija, inače null.
				\begin{longtblr}[
					label=none,
					entry=none
					]{
						width = \textwidth,
						colspec={|X[10,l]|X[6, l]|X[20, l]|}, 
						rowhead = 1,
					} %definicija širine tablice, širine stupaca, poravnanje i broja redaka naslova tablice
					\hline \multicolumn{3}{|c|}{\textbf{zdravstveniPodaci}}	 \\ \hline[3pt]
					\SetCell{LightGreen}idZdravstvenih & SERIAL & id zdravstvenog podatka \\ \hline
					zdravstveniPodatak & VARCHAR & opis zdravstvenog podatka koji se ispisuje, trebala bi biti izjava, koja ako je istinita, odbija donora privremeno ili trajno
					npr. osoba teži manje od 55 kg \\ \hline
					tezinaKriterija 	& INT &  tezina kriterija ( 0 za trajno, 1 za privremeno) 	\\ \hline 

					
				\end{longtblr}
\eject				
				\textbf{doniranjeZdravljeOdgovori\textit{•}}
				
				entitet koji služi kao spremište odgovora korisnika koji su došli donirati krv
				\begin{longtblr}[
					label=none,
					entry=none
					]{
						width = \textwidth,
						colspec={|X[10,l]|X[6, l]|X[20, l]|}, 
						rowhead = 1,
					} %definicija širine tablice, širine stupaca, poravnanje i broja redaka naslova tablice
					\hline \multicolumn{3}{|c|}{\textbf{doniranjeZdravljeOdgovori}}	 \\ \hline[3pt]
					\SetCell{LightGreen} brDoniranja & INT & brDoniranja iz tablice pokusajDoniranja \\ \hline
					\SetCell{LightGreen} idZdravstvenih & INT &  idZdravstvenih iz tablice zdravstveni podaci \\ \hline
					odgovorDonora & boolean & true ako donor pada po kriteriju i odbija se privremeno/trajno ,inače false\\ \hline
					
				\end{longtblr}
			
			\subsection{Dijagram baze podataka}
				\begin{figure}[H]
	\centering
	\includegraphics[width=\textwidth, scale=2.0]{dijagrami/relShema.png}
	\caption{relacijski model baze podataka}
	\label{fig:dijagram_baze}
\end{figure}
			\eject
			
		\section{Dijagram razreda}
		
	Dijagram razreda pokazuje odnose između različitih objekata, te njihove atribute i operacije kojima vladaju. Na slikama 4.5., 4.6, 4.7. prikazani su razredi koji pripadaju \textit{backend} dijelu naše arhitekture. Radi jednostavnosti, dijagram razreda je podijeljen u više slika, no bez obzira na to, prikazani razredi na neki način komuniciraju.
	Na idućoj slici prikazan je model podataka kojima backend rukuje. Korisnik aplikacije (administartor, djelatnik banke, donor) modeliran je razredom \textit{User}. Razred \textit{Blood} modelira krvne grupe. Razred \textit{Consumption} modelira potrošnju krvi. Razred \textit{Donation} modelira pokušaj donacije krvi, dok razred \textit{Role} modelira 3 moguće uloge u aplikaciji. Također navedene su enumeracije za nazive krvnih grupa i nazive uloga.
%\begin{figure}[H]
%	\centering
%	\includegraphics[width=\textwidth, scale=2.0]{dijagrami/}
%	\caption{Dijagram razreda koji opisuje model}
%	\label{fig:dijagram_modela}
%\end{figure}
		
	\eject
	Na idućoj slici prikazana je sredina \text{backenda}. Kao glavna komponenta na slici prikazano je sučelje JpaRepository, koje predstavlja apstraktni repozitorij podataka. Iz tog sučelja, izvedena su sučelja UserRepository, RoleRepository, BloodRepository, ConsumptionRepository, DonationRepository. Ta sučelja predstavljaju repozitorij podataka za prije navedene razrede modela, tj. oni predstavljaju poveznicu s bazom ili DAO (eng. Data Access Object). Također koriste se različite ServiceJpa klase koje koriste te repozitorije, te one implementiraju sučelje imeKlase_Service. 
	
%\begin{figure}[H]
%	\centering
%	\includegraphics[width=\textwidth, scale=2.0]{dijagrami/}
%	\caption{Dijagram razreda koji opisuju repozitorije}
%	\label{fig:dijagram_repozitorija}
%\end{figure}

\eject

	Na zadnjoj slici prikazan je prednji dio \text{backend} dijela koji je povezan sa stvarnim svijetom. Ovdje vidimo razrede UserController, RoleController, BloodController, ConsumptionController i DonationController. Svi ti razredi koriste Java anotaciju RestController koji predstavlja REST endpoint. Ti razredi su oni koji dobivaju zahtjeve iz vanjskog svijeta, a odgovaraju HTTP odgovorima i JSON objektima.
	
%\begin{figure}[H]
%	\centering
%	\includegraphics[width=\textwidth, scale=2.0]%{dijagrami/}
%	\caption{Dijagram razreda koji opisuju kontrolere}
%	\label{fig:dijagram_kontrolera}
%\end{figure}

\eject
			
			\textit{Prilikom prve predaje projekta, potrebno je priložiti potpuno razrađen dijagram razreda vezan uz \textbf{generičku funkcionalnost} sustava. Ostale funkcionalnosti trebaju biti idejno razrađene u dijagramu sa sljedećim komponentama: nazivi razreda, nazivi metoda i vrste pristupa metodama (npr. javni, zaštićeni), nazivi atributa razreda, veze i odnosi između razreda.}\\
			
			\textbf{\textit{dio 2. revizije}}\\			
			
			\textit{Prilikom druge predaje projekta dijagram razreda i opisi moraju odgovarati stvarnom stanju implementacije}
			
			
			
			\eject
		
		\section{Dijagram stanja}
			
			
			\textbf{\textit{dio 2. revizije}}\\
			
			\textit{Potrebno je priložiti dijagram stanja i opisati ga. Dovoljan je jedan dijagram stanja koji prikazuje \textbf{značajan dio funkcionalnosti} sustava. Na primjer, stanja korisničkog sučelja i tijek korištenja neke ključne funkcionalnosti jesu značajan dio sustava, a registracija i prijava nisu. }
			
			
			\eject 
		
		\section{Dijagram aktivnosti}
			
			\textbf{\textit{dio 2. revizije}}\\
			
			 \textit{Potrebno je priložiti dijagram aktivnosti s pripadajućim opisom. Dijagram aktivnosti treba prikazivati značajan dio sustava.}
			
			\eject
		\section{Dijagram komponenti}
		
			\textbf{\textit{dio 2. revizije}}\\
		
			 \textit{Potrebno je priložiti dijagram komponenti s pripadajućim opisom. Dijagram komponenti treba prikazivati strukturu cijele aplikacije.}
